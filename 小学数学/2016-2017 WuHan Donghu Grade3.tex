\documentclass[12pt,twoside,space]{ctexart}
\usepackage{NEMT}
\usepackage{float} %设置图片浮动位置的宏包
\begin{document}\zihao{5}
\juemi %输出绝密
\biaoti{2016-2017学年武汉市东湖高新区三年级(上)期末数学试卷}

  \section{认真审题,仔细填题(每空1分,共23分)}
    \begin{enumerate}[itemsep=-0.2em,topsep=0pt]
      \item 在里填上“$>$”,“$<$”或“$=$”.\\
          \begin{tabular*}{\textwidth}{@{\extracolsep{\fill}} lll}
            90秒 $\bigcirc$ 9分 & 1分15秒 $\bigcirc$ 65秒 & $\dfrac{1}{4} \bigcirc \dfrac{1}{8}$ \\
            6厘米 $\bigcirc$ 60毫米 & 600千克 $\bigcirc$ 6吨 & $1 \bigcirc \dfrac{7}{7}$
          \end{tabular*}

        \item 1分40秒= \underline{\hspace{3em}}秒\\
        3600千克+400千克= \underline{\hspace{3em}}吨

        \item 把一个圆对折再对折后,每份是它的 $\dfrac{(\hspace{1em})}{(\hspace{1em})}$,读作 \underline{\hspace{2em}}分之 \underline{\hspace{2em}} .

        \item 某同学的身份证号码是420107204902224368.根据这个身份证号,可以知道这位同学的生日是 \underline{\hspace{2em}} 年 \underline{\hspace{2em}} 月 \underline{\hspace{2em}} 日,性别是 \underline{\hspace{2em}}.
    \end{enumerate}

    \section{选择题(每小题2分,满分10分)}
      \begin{enumerate}[itemsep=-0.2em,topsep=0pt]
        \item 用两根同样长的铁丝可以围成多种长、宽不相同的长方形,这些长方形的周长(  )
          \begin{tasks}(3)
            \task 相等	\task 不相等	\task 不能确定
          \end{tasks}

        \item 小明8时50分到电影院时,电影还有20分钟放映,电影是(  )开始的.
          \begin{tasks}(3)
            \task $9:10$	\task $8:50$	\task $8:30$
          \end{tasks}
        
        \item $\Box 96$是一个三位数,$\Box 96\times 5$的积最接近2000,$\Box$里数字是(  )
          \begin{tasks}(3)
            \task 3	\task 4	\task 5
          \end{tasks}

        \item 小明吃了苹果的$\dfrac{1}{2}$,小兰吃了桃子的 $\dfrac{1}{2}$,那么(  )
          \begin{tasks}(3)
            \task 吃得一样多	\task 小明吃得多	\task 小兰吃得多	\task 无法确定
          \end{tasks}
      \end{enumerate}

    \section{计算题(满分18分)}
      \begin{enumerate}[itemsep=-0.2em,topsep=0pt]
        \item 列竖式计算下面各题(带 \ding{72} 号要验算).\\
            \begin{tabular*}{\textwidth}{@{\extracolsep{\fill}} llll}
              $532\times 8=$ \vspace{3em} & $609\times 5=$ & \ding{72} $599+608=$ \\
              \ding{72} $1000-724=$ \vspace{1em} & \ding{72} $940-59=$ & $234\times 9=$
            \end{tabular*}
      \end{enumerate}
    
    \newpage
    \section{实验操作题(满分12分)}
      \begin{enumerate}[itemsep=0.2em, topsep=0pt]
        \item 表示分数 $\dfrac{1}{3}$.\\[0.5em]
            \begin{tabular}{|l|l|l|l|l|l|l|l|l|l|l|l|}
            \hline
              \hspace{1em} & \hspace{1em} & \hspace{1em} & \hspace{1em} & \hspace{1em} & \hspace{1em} & \hspace{1em} & \hspace{1em} & \hspace{1em} & \hspace{1em} &  \hspace{1em} & \hspace{1em} \\ \hline
            \end{tabular}

          \item 在下面方格图中画一个周长是10厘米的长方形.(每个小格子的边长都是1厘米)
            \begin{figure}[H]
              \centering
              \includegraphics[width=0.35\textwidth]{./wh19.png}
            \end{figure}
      \end{enumerate}

    \section{走进生活,解决问题(共6小题,满分34分)}
      \begin{enumerate}[itemsep=4em,topsep=0pt]
        \item 科技园内上午有游客892人,中午有265人离开.下午又来了403位游客,这时园内有多少游客?园内全天来了多少位游客?
        \item 军棋的价钱是9元,象棋的价钱是军棋的5倍,象棋的价钱是多少元?
        \item 李叔叔加工一批零件,4小时完成了32个,照这样的速度,8小时能加工多少个?
        \item 图书角有18本图书,其中$\dfrac{1}{3}$是科技书,科技书有多少本?
        \item 师徒二人一起加工一批零件,师傅每小时加工3个零件,徒弟侮小时加工2个零件,怎样安排能恰好加工完8个零件?
        \item 一块长方形菜地,长15米,宽7米,现在要围上篱笆,一面靠墙,至少要多少篱笆?
      \end{enumerate}
\end{document}