\documentclass[12pt,twoside,space]{ctexart}
\usepackage{NEMT}
\usepackage{float} %设置图片浮动位置的宏包
\usepackage{wasysym}

\begin{document}\zihao{5}
\juemi %输出绝密
\biaoti{2017-2018学年南京市浦口区三年级(上)期末数学试卷}

  \section{填空题(每空1分)}
    \begin{enumerate}[itemsep=-0.2em,topsep=0pt]
      \item 口算$300\div 3$,可以看作 \underline{\hspace{2em}} 个 \underline{\hspace{2em}} 除以3;而口算$300\div 5$,可以看作 \underline{\hspace{2em}} 个 \underline{\hspace{2em}} 除以5.

        \item $2\Box 9\times 4$,要使它的积是四位数,$\Box$里最小填 \underline{\hspace{2em}}.$8\Box 5\div 4$,要使商的十位是0,$\Box$里最大可以填 \underline{\hspace{2em}}.如果$85\div \Box$的商是二十多,$\Box$里可能是 \underline{\hspace{2em}}.

        \item 在横线上填合适的单位.
          \begin{enumerate}[align=left,labelsep=-0.6em,leftmargin=1.2em,noitemsep,topsep=0pt,label={(\arabic*)}]
            \item 5枚1元的硬币大约重30 \underline{\hspace{2em}}.
            \item 数学书封面的周长大约是90 \underline{\hspace{2em}} .
            \item 乐乐今年上三年级,体重为32 \underline{\hspace{2em}}.
          \end{enumerate}
    \end{enumerate}

    \section{选择题(每小题2分,满分10分)}
      \begin{enumerate}[itemsep=-0.2em,topsep=0pt]
        \item 用两根同样长的铁丝可以围成多种长、宽不相同的长方形,这些长方形的周长(  )
          \begin{tasks}(3)
            \task 相等	\task 不相等	\task 不能确定
          \end{tasks}

        \item 小明8时50分到电影院时,电影还有20分钟放映,电影是(  )开始的.
          \begin{tasks}(3)
            \task $9:10$	\task $8:50$	\task $8:30$
          \end{tasks}
        
        \item $\Box 96$是一个三位数,$\Box 96\times 5$的积最接近2000,$\Box$里数字是(  )
          \begin{tasks}(3)
            \task 3	\task 4	\task 5
          \end{tasks}

        \item 小明吃了苹果的$\dfrac{1}{2}$,小兰吃了桃子的 $\dfrac{1}{2}$,那么(  )
          \begin{tasks}(3)
            \task 吃得一样多	\task 小明吃得多	\task 小兰吃得多	\task 无法确定
          \end{tasks}
      \end{enumerate}

    \section{计算(共2小题,27分)}
      \begin{enumerate}[itemsep=-0.2em,topsep=0pt]
        \item 直接写出得数.\\
          {
            \def\arraystretch{1.6}% 
            \begin{tabular*}{\textwidth}{@{\extracolsep{\fill}} llll}
              $5\times 800=$ & $63\div 3=$ & $12\times 4=$ & $\dfrac{1}{6}-\dfrac{4}{6}=$ \\
              $420\div 6=$ & $3\times 17=$ & $52\div 2=$ & $\dfrac{4}{7} -\dfrac{2}{7}=$
            \end{tabular*}
          }

        \item 列竖式计算下面各题(带 \ding{72} 号要验算).\\[0.2em]
          \begin{tabular*}{\textwidth}{@{\extracolsep{\fill}} llll}
            $6\times 350=$ \vspace{3em} & $736\times 8=$ & $409\times 9=$ \\
            $601\div 3=$ \vspace{1em} & $700\div 5=$ & \ding{72} $256\div 6=$
          \end{tabular*}
      \end{enumerate}
    
    \section{操作题}
      \begin{enumerate}[itemsep=0.2em, topsep=0pt]
        \item 下边方格图中 $\blacksquare$ 占一个图形的$\dfrac{1}{6}$,请你在方格图中画出这个图形.
          \begin{figure}[H]
            \centering
            \includegraphics[width=0.25\textwidth]{./nj26.png}
          \end{figure}

          \item 小熊吃了一个西瓜的 $\dfrac{1}{2}$,小猴子吃了一个西瓜的$\dfrac{1}{3}$,结果小猴子吃的西瓜比小熊多.请解释为什么?可以写一写,画一画,让人一看就明白.
      \end{enumerate}

    \section{走进生活,解决问题(满分34分)}
      \begin{enumerate}[itemsep=1em,topsep=0pt]
        \item 一块地的$\dfrac{2}{7}$种青菜,$\dfrac{3}{7}$种菠菜.种这两种蔬菜的地一共占这块地的几分之几?还剩这块地的几分之几?
        \item 看图,列算式并回答问题。
          \begin{enumerate}[align=left,labelsep=-0.6em,leftmargin=1.2em,noitemsep,topsep=0pt,label={(\arabic*)}]
            \item 今年爷爷的年龄是小明的几倍?
            \item  妈妈今年多少岁?
          \end{enumerate}
          \begin{minipage}{\linewidth}
            \includegraphics[width=0.55\textwidth]{./nj29.png}
          \end{minipage}

        \item 赵华家今年收获的核桃和红枣一共51袋.卖掉15袋核桃后,剩下的核桃和红枣袋数相等.他家今年收获的红枣和核桃各有多少袋?
        \item 如图中有一些数学信息,你能把它们编成一个数学故事吗?请写出来,并解答其中的数学问题.
          \begin{enumerate}[align=left,labelsep=-0.6em,leftmargin=1.2em,noitemsep,topsep=0pt,label={(\arabic*)}]
            \item 数学故事:\underline{\hspace{28em}}
            \item 解答:\underline{\hspace{28em}}
          \end{enumerate}

        \item 三年级同学参加方队表演,原来每行站16人,正好站成6行.现改变方队的队形,站成8行.
          \begin{enumerate}[align=left,labelsep=-0.6em,leftmargin=1.2em,noitemsep,topsep=0pt,label={(\arabic*)}]
            \item 现在的方队每行站几人?
            \item  如果这些同学按现在的队形站立,最外圈同学穿黄色运动服,其余同学穿红色运动服,那么方队中穿黄色运动服的有多少人?(可以用一个“$\CIRCLE$”表示1个人来排一排、想一想哦!)
          \end{enumerate}
      \end{enumerate}
\end{document}