\documentclass[12pt,twoside,space]{ctexart}
\usepackage{NEMT}
\usepackage{float} %设置图片浮动位置的宏包
\usepackage{siunitx}

\begin{document}\zihao{5}
\juemi %输出绝密
\biaoti{2017-2018学年武汉市东湖新区六年级(上)期末数学试卷}

  \section{填空题(每空1分)}
    \begin{enumerate}[itemsep=0.2em,topsep=0pt]
      \item 把$3:0.25$化成最简单的整数比是 $\underline{\hspace{2em}}: \underline{\hspace{2em}}$ ,它的比值是 \underline{\hspace{3em}}.
      \item $\dfrac{(\hspace{1em})}{(\hspace{1em})}=0.375=\underline{\hspace{3em}}:24=24:\underline{\hspace{3em}}=\underline{\hspace{3em}}\%$.
      \item  \underline{\hspace{3em}}米的$\dfrac{4}{7}$是28米,比60千克轻10\%的是 \underline{\hspace{3em}} 千克.
      \item 用一根长18.84分米的铁丝围成一个圆,这个圆的面积是 \underline{\hspace{3em}} 平方分米.
    \end{enumerate}

  \section{计算题(32分)}
    \begin{enumerate}[itemsep=0.2em,topsep=0pt]
      \item 直接写出得数.\\
        {
          \def\arraystretch{1.8}% 
          \begin{tabular*}{\textwidth}{@{\extracolsep{\fill}} lllll}
            $\dfrac{5}{9}\times 5=$ & $24 \div \dfrac{3}{8}=$ & $\dfrac{1}{3} + \dfrac{3}{4}=$ & $\dfrac{3}{10}\times \dfrac{5}{7}=$ & $\dfrac{3}{22}\div \dfrac{3}{16}=$ \\
            $5.4\times \dfrac{7}{9}=$ & $\dfrac{4}{5}\times \dfrac{10}{27}=$ & $1\div \dfrac{5}{12}=$ & $\dfrac{7}{12}\div 87.5\%=$ & $5^2-4^2=$
          \end{tabular*}
        }

      \item 下面各题怎样简便就怎样算.\\[0.5em]
          \begin{tabular*}{\textwidth}{@{\extracolsep{\fill}} llll}
            $\dfrac{8}{9}\times \dfrac{3}{4} + \dfrac{1}{27}$ \vspace{5em} & $\dfrac{5}{9}+\dfrac{4}{9}+ \dfrac{1}{3}$ & $(\dfrac{3}{5} - \dfrac{11}{20})\times 40$\\
            $\dfrac{4}{15}\times \dfrac{2}{7} + \dfrac{4}{15}\times \dfrac{2}{7}$ \vspace{3em} & $\dfrac{17}{24}\times 0.75+\dfrac{17}{24}\times 4$ & $\dfrac{15}{16}+(\dfrac{7}{16} -\dfrac{1}{4})+\dfrac{1}{2}$
          \end{tabular*}

      \item 解方程.\\[0.5em]
          \begin{tabular*}{\textwidth}{@{\extracolsep{\fill}} ll}
            $45\div x=\dfrac{9}{14}$ \vspace{3em} & $x-\dfrac{5}{6}x=1.2\times 5$ \\
            $4x+\dfrac{3}{8}=\dfrac{11}{16}$ \vspace{3em}
          \end{tabular*}
    \end{enumerate}

    \section{解决问题.(30分)}
      \begin{enumerate}[itemsep=4em,topsep=0pt]
        \item “狗年”就要到了.玩具厂要赶制一批“狗娃娃”,甲组单独生产需要15天,乙组单独生产需要10天.现在两个组合作,多少天能完成任务?
        \item 五年级同学收集了360个饮料瓶,六年级同学比五年级多收集了20\%,六年级同学收集了多少个饮料瓶?
        \item 去年全国电商“双12”促销活动共产生2.43亿个快递件,前3天时间送达了 $\dfrac{2}{5}$,照这样的速度,剩下的快递件还需要几天送达?
        \item 工厂计划加工一批零件,己加工的与未加工的个数比是$3:2$,如果再加工260个,就会超过计划的12\%.计划完成多少个?还需要加工多少个才完成任务?
        \item 把两份白糖分别倒入两杯白开水中,得到两杯糖水(都能完全溶解).
          \begin{enumerate}[align=left,labelsep=-0.6em,leftmargin=1.2em,noitemsep,topsep=0pt,label={(\arabic*)}]
            \item 选择 \underline{\hspace{2em}} 杯白糖和 \underline{\hspace{2em}} 杯白开水,糖水最甜.
            \item 最甜糖水的含糖率是多少?\\[0.5em]
              \begin{tabular}{|l|l|l|}
              \hline
              白糖(相同品牌) & A杯210克 & B杯30克 \\ \hline
              白开水(水温相同) & C杯210克 & D杯300克 \\ \hline
              \end{tabular}

          \end{enumerate}
      \end{enumerate}
\end{document}